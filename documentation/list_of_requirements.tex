\section{List of Requirements}
Requirement specification 
The overall goal of the project is to simulate
\begin{itemize}
  \item a) the human body with the illness diabetes and the aspects introduced
  by this illness. As well as the reaction on food and insulin injection.
  \item b) an insulin pump which reacts on the behavior of the the above given
  simulation of the human body with the illness diabetes.
\end{itemize}
The first step therefore is to provide a simulation of the human body with the
illness diabetes and the aspects food and insulin. 
Since this is quite a big task it is divided into modules. These modules should
first on their own simulate the behavior of diabetes, food and insulin and are
therefore called behavioral modules.

\subsection{Behavioral Modules}

\subsubsection{Diabetes Module}

\subsubsection{Food Module}
For reasons of simplification only the three major food groups are taken into
account.
These groups are 
\begin{itemize}
  \item high glycemic (fast),
  \item moderate glycemic (intermediate) and
  \item low glycemic (slow) foods.
\end{itemize}
In order to provide means to calculate the behavior of these groups
mathematical models need to be developed and / or researched to allow the
simulation.

\subsubsection{Insulin Module}
There are many insulin types, because of the given possibility to mix different 
insulin types together. Therefore we decided to implement the following three types of insulin:
\begin{itemize}
   \item Rapid-Acting
   \item Short-Acting
   \item Long-Acting
\end{itemize}
The different properties of each type are described in chapter 2.1.1.3!\\
DESCRIPTION OF THE MATHEMATICAL MODEL!\\

\subsection{Body Simulation}
The body simulation combines all behavioral modules and therefore simulates the
behavior of the human body with the illness diabetes.
Inputs of this body simulation are food and insulin.
Outputs are glucose and insulin values over time.

Concrete requirements are as follows:
\begin{itemize}
  \item Inputs must be realized as GUI and as inputs via given CSV-Files.
  \item Outputs must be realized as GUI and as CSV-Files.
  \item The behavioral modules created in the first step are combined to one
  large model.
\end{itemize} 

\subsection{Test Cases}
Test cases need to be designed for the complete simulation but as well for the
smaller parts (e.g. behavioral modules etc.).
